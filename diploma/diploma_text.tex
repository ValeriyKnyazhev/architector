\documentclass[a4paper,14pt]{extreport} %размер бумаги устанавливаем А4, шрифт 14пунктов
\usepackage[T2A]{fontenc}
\usepackage[utf8]{inputenc}%включаем свою кодировку: koi8-r или utf8 в UNIX, cp1251 в Windows
\usepackage[english,russian]{babel}%используем русский и английский языки с переносами
\usepackage{amssymb,amsfonts,amsmath,mathtext,cite,enumerate,float} %подключаем нужные пакеты расширений
\usepackage[dvips]{graphicx}
\usepackage{pgfplots}
%\usepackage{listings}
\usepackage[linesnumbered,boxed]{algorithm2e}
\graphicspath{{./images/}}%путь к рисункам


\makeatletter
\renewcommand{\@biblabel}[1]{#1.} % Заменяем библиографию с квадратных скобок на точку:
\makeatother

\usepackage{extsizes}

\usepackage{geometry} % Меняем поля страницы
\geometry{left=3cm}% левое поле
\geometry{right=2cm}% правое поле
\geometry{top=2cm}% верхнее поле
\geometry{bottom=2cm}% нижнее поле
\linespread{1.3}

%\renewcommand{\theenumi}{\arabic{enumi}}% Меняем везде перечисления на цифра.цифра
%\renewcommand{\labelenumi}{\arabic{enumi}}% Меняем везде перечисления на цифра.цифра
%\renewcommand{\theenumii}{.\arabic{enumii}}% Меняем везде перечисления на цифра.цифра
%\renewcommand{\labelenumii}{\arabic{enumi}.\arabic{enumii}.}% Меняем везде перечисления на цифра.цифра
%\renewcommand{\theenumiii}{.\arabic{enumiii}}% Меняем везде перечисления на цифра.цифра
%\renewcommand{\labelenumiii}{\arabic{enumi}.\arabic{enumii}.\arabic{enumiii}.}% Меняем везде перечисления на цифра.цифра


\begin{document}
\begin{titlepage}
\newpage

\begin{center}
\small МИНИСТЕРСТВО ОБРАЗОВАНИЯ И НАУКИ РОССИЙСКОЙ ФЕДЕРАЦИИ \\
\vspace{1cm}
\small ФЕДЕРАЛЬНОЕ ГОСУДАРСТВЕННОЕ АВТОНОМНОЕ ОБРАЗОВАТЕЛЬНОЕ \\*
\small УЧРЕЖДЕНИЕ ВЫСШЕГО ОБРАЗОВАНИЯ \\*
\small "МОСКОВСКИЙ ФИЗИКО-ТЕХНИЧЕСКИЙ ИНСТИТУТ \\*
\small (ГОСУДАРСТВЕННЫЙ УНИВЕРСИТЕТ)" \\*
\vspace{1cm}
\small ФАКУЛЬТЕТ ИННОВАЦИЙ И ВЫСОКИХ ТЕХНОЛОГИЙ \\*
\small КАФЕДРА ТЕОРЕТИЧЕСКИХ И ПРИКЛАДНЫХ ПРОБЛЕМ ИННОВАЦИЙ \\*
\hrulefill
\end{center}

\vspace{4em}

\begin{center}
\textbf{ВЫПУСКНАЯ КВАЛИФИКАЦИОННАЯ РАБОТА} \\
\vspace{1em}
\small \textbf{(МАГИСТЕРСКАЯ РАБОТА)} \\
\vspace{1em}
\small \textbf{Направление подготовки: 03.04.01 "Прикладные математика и физика"} \\
\vspace{1em}
\textsc{\textbf{НА ТЕМУ:}} \\
\vspace{2em}
\large \textsc{\textbf{Единая автоматизированная информационная система поддержки и сопровождения проектов, созданных с применением стандарта BIM}}
\end{center}

\vspace{6em}

\begin{flushleft}
Студент \hrulefill Княжев В.А. \\
\vspace{1em}
Научный руководитель \hrulefill Зырин С.В.\\
\vspace{1em}
%Рецензент \\
%к.ф.-м.н., в.н.с. АБВГ \hrulefill Петров В.В.\\
%\vspace{1.5em}
%Зав. кафедрой  ХХХ \\
%д.ф-м.н, профессор \hrulefill Сидоров Г.Г.
\end{flushleft}

\vspace{\fill}

\begin{center}
г. Москва, 2019
\end{center}

\end{titlepage}% это титульный лист
\tableofcontents % это оглавление, которое генерируется автоматически


\chapter{Введение}
\section{Актуальность проблемы}

\newpage
\section{Постановка задачи}

\newpage
\section{План работ}

\newpage

\chapter{Документ о концепции и границах}
\section{Бизнес-требования}
\subsection{Исходные данные}

На текущий момент архитекторам требуется иметь веб-платформу для управления своими проектами. А именно от платформы требуется предоставить возможности создания, хранения, изменения архитектурных проектов, а также возможности предоставления  доступа другим пользователям и просмотр истории изменений проектов.


\subsection{Возможности бизнеса}

Ныне существующие системы не позволяют редактировать составные части проектов разными людьми в одно и то же время, а также не дают возможности просматривать хронологию изменений продукта.


\subsection{Бизнес-цели}

\begin{table}
\caption {Нефинансовые цели} \label{tab:title}
\begin{center}
\begin{tabular}{ | l | p{14cm} | }
\hline
№ & Цель \\
\hline
Н1 & Разработать веб-платформу для управления и редактирования 	архитектурных проектов \\
\hline
Н2 & Разработать алгоритм вычисления различий (diff) между файлами в разные промежутки времени \\
\hline
Н3 & Разработать алгоритм объединения (merge) различных версий файлов при наличии конфликтов между ними \\
\hline
Н4 & Реализовать алгоритм визуализации хронологических изменений проектов \\
\hline
\end{tabular}
\end{center}
\end{table}
 
 
\subsection{Критерии успеха}

\begin{itemize}
\item Веб-платформа позволяет создать, изменить, удалить архитектурный проект
\item В веб-платформе имеется возможность просмотреть список изменений между различными версиями (временными) проекта: список файлов со списком строк с изменениями
\item При наличии конфликтов во время изменения какой-либо части проекта имеется возможность выбрать или заново написать в каждом блоке изменений правильный вариант кода, который будет сохранен после подтверждения редактором
\end {itemize}
 
 
 \subsection{Положение о концепции проекта}
 
Для архитекторов, которым требуется управлять своими архитектурными проектами, а также иметь возможность отслеживать изменения проекта во времени, Architector является веб-платформой, которая будет выступать в качестве единой системы по хранению и изменению архитектурных проектов. В отличии от других существующих систем Architector позволит  просматривать изменения проекта во времени, а также даст возможность вносить конфликтующие изменения в проекты.


\subsection{Бизнес-риски}
 
\begin{itemize}
\item Сложности при реализации алгоритма нахождения различий между версиями проекта
\item Корректность объединения изменений, выполненных  в примерно один и тот же малый отрезок времени
\end {itemize}

\newpage

\section{Рамки и ограничения проекта}
\subsection{Основные функции}

\begin{enumerate}
\item Создание проекта
\item Изменение описательных данных(?) в проекте
\item Добавление в проект файлов с контентом, на основе которых и будет строиться конкретное решение архитектора
\item Изменение описательных данных(?) в файлах проекта
\item Просмотр контента текущей версии файла
\item Изменение контента файла
\item Просмотр списка изменений между различными версиями проекта
\item Отображение контента проекта или файла в определенный момент времени в истории
\item Отображение списка изменений описательных данных и контента проекта и файлов в каждой отдельно взятой единице записей изменений проекта
\item Просмотр конфликтов при изменении проекта
\item Разрешение конфликта при сохранении изменений в проект
\end {enumerate}

\subsection{Ограничения и исключения}

\begin{itemize}
\item Размер каждого файла должен не превышать 150 Мб (ограничение ifc формата)
\end {itemize}

\newpage

\chapter{Функции системы}
\section{Основные требования к платформе}

\begin{enumerate}
\item Создание проекта \\
\begin{tabular}{ | p{2cm} | p{12cm} | }
\hline
\multicolumn{2}{ | c | }{Создание проекта с указанием названия, описания, автора, а также указанием прав доступа} \\
\hline
\multicolumn{2}{ | l | }{Функциональные требования:} \\
\hline
СПФ1 & При создании проекта система должна предоставить пользователю идентификатор, по которому он теперь сможет работать с только что созданным проектом \\
\hline
СПФ2 & При создании проекта система дает возможность указать права доступа для других пользователей \\
\hline
\multicolumn{2}{ | l | }{Нефункциональные требования:} \\
\hline
СПН1 & Права пользователей подразделяются на чтение, редактирование. Права могут выдаваться как только автору проекта, так и списку пользователей, которым данное разрешение предоставит автор \\
\hline
\end{tabular}
\item Изменение описательных данных(?) в проекте
\item Добавление в проект файлов с контентом, на основе которых и будет строиться конкретное решение архитектора
\item Изменение описательных данных(?) в файлах проекта
\item Просмотр контента текущей версии файла
\item Изменение контента файла
\item Просмотр списка изменений между различными версиями проекта
\item Отображение контента проекта или файла в определенный момент времени в истории
\item Отображение списка изменений описательных данных и контента проекта и файлов в каждой отдельно взятой единице записей изменений проекта
\item Просмотр конфликтов при изменении проекта
\item Разрешение конфликта при сохранении изменений в проект
\end {enumerate}


\begin{thebibliography}{1}
{\small
\bibitem{BIB_EXAMPLE} {\it Author1, Author2.}
\textbf{The name of example} // conference of this article. 2019. pp. 45-49
}
\end{thebibliography}

\end{document}