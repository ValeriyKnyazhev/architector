\documentclass[a4paper,14pt]{extreport} %размер бумаги устанавливаем А4, шрифт 14пунктов
\usepackage[T2A]{fontenc}
\usepackage{booktabs} % For prettier tables
\usepackage[utf8]{inputenc}%включаем свою кодировку: koi8-r или utf8 в UNIX, cp1251 в Windows
\usepackage[english,russian]{babel}%используем русский и английский языки с переносами
\usepackage{amssymb,amsfonts,amsmath,mathtext,cite,enumerate,float} %подключаем нужные пакеты расширений
\usepackage[dvips]{graphicx}
\usepackage{pgfplots}
%\usepackage{listings}
\usepackage[linesnumbered,boxed]{algorithm2e}
\graphicspath{{./images/}}%путь к рисункам


\makeatletter
\renewcommand{\@biblabel}[1]{#1.} % Заменяем библиографию с квадратных скобок на точку:
\makeatother

\usepackage{extsizes}

\usepackage{geometry} % Меняем поля страницы
\geometry{left=3cm}% левое поле
\geometry{right=2cm}% правое поле
\geometry{top=2cm}% верхнее поле
\geometry{bottom=2cm}% нижнее поле
\linespread{1.3}

%\renewcommand{\theenumi}{\arabic{enumi}}% Меняем везде перечисления на цифра.цифра
%\renewcommand{\labelenumi}{\arabic{enumi}}% Меняем везде перечисления на цифра.цифра
%\renewcommand{\theenumii}{.\arabic{enumii}}% Меняем везде перечисления на цифра.цифра
%\renewcommand{\labelenumii}{\arabic{enumi}.\arabic{enumii}.}% Меняем везде перечисления на цифра.цифра
%\renewcommand{\theenumiii}{.\arabic{enumiii}}% Меняем везде перечисления на цифра.цифра
%\renewcommand{\labelenumiii}{\arabic{enumi}.\arabic{enumii}.\arabic{enumiii}.}% Меняем везде перечисления на цифра.цифра


\begin{document}
\begin{titlepage}
\newpage

\begin{center}
\small МИНИСТЕРСТВО ОБРАЗОВАНИЯ И НАУКИ РОССИЙСКОЙ ФЕДЕРАЦИИ \\
\vspace{1cm}
\small ФЕДЕРАЛЬНОЕ ГОСУДАРСТВЕННОЕ АВТОНОМНОЕ ОБРАЗОВАТЕЛЬНОЕ \\*
\small УЧРЕЖДЕНИЕ ВЫСШЕГО ОБРАЗОВАНИЯ \\*
\small "МОСКОВСКИЙ ФИЗИКО-ТЕХНИЧЕСКИЙ ИНСТИТУТ \\*
\small (ГОСУДАРСТВЕННЫЙ УНИВЕРСИТЕТ)" \\*
\vspace{1cm}
\small ФАКУЛЬТЕТ ИННОВАЦИЙ И ВЫСОКИХ ТЕХНОЛОГИЙ \\*
\small КАФЕДРА ТЕОРЕТИЧЕСКИХ И ПРИКЛАДНЫХ ПРОБЛЕМ ИННОВАЦИЙ \\*
\hrulefill
\end{center}

\vspace{4em}

\begin{center}
\textbf{ВЫПУСКНАЯ КВАЛИФИКАЦИОННАЯ РАБОТА} \\
\vspace{1em}
\small \textbf{(МАГИСТЕРСКАЯ РАБОТА)} \\
\vspace{1em}
\small \textbf{Направление подготовки: 03.04.01 "Прикладные математика и физика"} \\
\vspace{1em}
\textsc{\textbf{НА ТЕМУ:}} \\
\vspace{2em}
\large \textsc{\textbf{Единая автоматизированная информационная система поддержки и сопровождения проектов, созданных с применением стандарта BIM}}
\end{center}

\vspace{6em}

\begin{flushleft}
Студент \hrulefill Княжев В.А. \\
\vspace{1em}
Научный руководитель \hrulefill Зырин С.В.\\
\vspace{1em}
%Рецензент \\
%к.ф.-м.н., в.н.с. АБВГ \hrulefill Петров В.В.\\
%\vspace{1.5em}
%Зав. кафедрой  ХХХ \\
%д.ф-м.н, профессор \hrulefill Сидоров Г.Г.
\end{flushleft}

\vspace{\fill}

\begin{center}
г. Москва, 2019
\end{center}

\end{titlepage}% это титульный лист
\tableofcontents % это оглавление, которое генерируется автоматически


\chapter{Введение}
\section{Актуальность проблемы}
Темпы строительства зданий и промышленных объектов в мире и сложность конструкций увеличивается с каждым годом. Ранее использовавшиеся методы проектирования чертежей на бумаге отходят на второй план, и все более активно используются компьютерные технологии, а также становится очевидной необходимость повсеместного введения стандартов проектирования зданий. \\
Одним из наиболее современных стандартов проектирования является стандарт BIM (Building Information Modeling). Его концепция позволяет не только проектировать здания, но также охватить весь их жизненный цикл: от управления затратами и строительством здания до его эксплуатации. \\
Подобная всеобъемлемость хороша тем, что вся информация о конструкции содержится в одном проекте. Это помогает сохранять ее надежность, позволяет быстрее выявлять ошибки и уменьшать стоимость ремонта. Но также из этого вытекает необходимость координации одновременной работы большого количества людей над одним проектом: крупных команд архитекторов, иногда распределенных по всему миру, эксплуатирующих организаций и всех других людей, участвующих в  поддержании здания. \\
Поэтому очень важно иметь возможность одновременного изменения проекта здания разными людьми без потери каких-либо данных. Но малейшая ошибка в одном из элементов конструкции может быть критичной, поэтому важно в каждый момент времени знать, каким из участников и когда были внесены изменения в проект, чтобы все участники процесса несли индивидуальную ответственность за свою работу. \\
В настоящий момент программ, специализирующихся на архитектурных проектах, и  которые бы хорошо решали задачу по координации работы большого количества людей и отслеживания изменений, не существует. 

\newpage
\section{Постановка задачи}
Требуется разработать веб-платформу, которая будет предоставлять следующие возможности:
\begin{enumerate}
\item Управление жизненным циклом проектов. \\
Создание проекта, добавление, редактирование  и удаление файлов, управление правами доступа к проекту.
\item Отслеживание изменений проекта во времени. \\
Отображение списка всех изменений проекта, а также возможность просмотра версии данных или внесенных в проект изменений в конкретный момент времени.
\item Одновременное внесение изменений в проекты несколькими пользователями. \\
Пользователи могут работать над разными частями проекта в одно и то же время. При наличии конфликтующих изменений предоставляется возможность сохранения изменений, внесенных как другими пользователями, так и текущим.
\item Подготовка окружения, запуск системы и ее масштабируемость. \\
Возможность быстрой подготовки окружения и запуска сервиса для мговенного развертывания веб-платформы. В моменты пиковой нагрузки пользователей, веб-платформа не должна терять производительность.
\end{enumerate}

\newpage
\section{План работ}

\newpage

\chapter{Документ о концепции и границах}
\section{Бизнес-требования}
\subsection{Исходные данные}

На текущий момент архитекторам требуется иметь веб-платформу для управления своими проектами. А именно от платформы требуется предоставить возможности создания, хранения, изменения архитектурных проектов, а также возможности предоставления  доступа другим пользователям и просмотр истории изменений проектов.


\subsection{Возможности бизнеса}

Ныне существующие системы не позволяют редактировать составные части проектов разными людьми в одно и то же время, а также не дают возможности просматривать хронологию изменений продукта.

\newpage

\subsection{Бизнес-цели}

\begin{table}
\caption {Нефинансовые цели} \label{tab:title}
\begin{center}
\begin{tabular}{ | l | p{14cm} | }
\hline
№ & Цель \\
\hline
Н1 & Разработать веб-платформу для управления и редактирования 	архитектурных проектов \\
\hline
Н2 & Разработать алгоритм вычисления различий (diff) между файлами в разные промежутки времени \\
\hline
Н3 & Разработать алгоритм объединения (merge) различных версий файлов при наличии конфликтов между ними \\
\hline
Н4 & Реализовать алгоритм визуализации хронологических изменений проектов \\
\hline
\end{tabular}
\end{center}
\end{table}
 
 
\subsection{Критерии успеха}

\begin{itemize}
\item Веб-платформа позволяет создать, изменить, удалить архитектурный проект
\item В веб-платформе имеется возможность просмотреть список изменений между различными версиями (временными) проекта: список файлов со списком строк с изменениями
\item При наличии конфликтов во время изменения какой-либо части проекта имеется возможность выбрать или заново написать в каждом блоке изменений правильный вариант кода, который будет сохранен после подтверждения редактором
\end {itemize}
 
 
 \subsection{Положение о концепции проекта}
 
Для архитекторов, которым требуется управлять своими архитектурными проектами, а также иметь возможность отслеживать изменения проекта во времени, Architector является веб-платформой, которая будет выступать в качестве единой системы по хранению и изменению архитектурных проектов. В отличии от других существующих систем Architector позволит  просматривать изменения проекта во времени, а также даст возможность вносить конфликтующие изменения в проекты.


\subsection{Бизнес-риски}
 
\begin{itemize}
\item Сложности при реализации алгоритма нахождения различий между версиями проекта
\item Корректность объединения изменений, выполненных  в примерно один и тот же малый отрезок времени
\end {itemize}

\newpage

\section{Рамки и ограничения проекта}
\subsection{Основные функции}

\begin{enumerate}
\item Создание проекта
\item Просмотр списка существующих и доступных пользователю проектов
\item Изменение описательных данных(?) в проекте
\item Добавление в проект файлов с контентом, на основе которых и будет строиться конкретное решение архитектора
\item Изменение описательных данных(?) в файлах проекта
\item Просмотр контента текущей версии файла
\item Изменение контента файла
\item Просмотр списка изменений между различными версиями проекта
\item Отображение контента проекта или файла в определенный момент времени в истории
\item Отображение списка изменений описательных данных и контента проекта и файлов в каждой отдельно взятой единице записей изменений проекта
\item Просмотр конфликтов при изменении проекта
\item Разрешение конфликта при сохранении изменений в проект
\end {enumerate}

\subsection{Ограничения и исключения}

\begin{itemize}
\item Размер каждого файла должен не превышать 150 Мб (ограничение ifc формата)
\end {itemize}

\newpage

\chapter{Функции системы}
\section{Основные требования к платформе}

\begin{enumerate}

\item Просмотр списка проектов \\

\begin{tabular}{| p{2.5cm}  | p{11.5cm} |}
\hline
Описание & Пользователь может просмотреть список доступных ему преоктов. Также для поиска проектов имеется возможность фильтрации данных. \\
\hline
\multicolumn{2}{ | l | }{Функциональные требования:} \\
\hline
ПСПФ1 & Система должна предоставить список всех проектов по заданным фильтрам  \\
\hline
ПСПФ2 & Записи проектов должны содержать следующую информацию: имя, описание проекта, даты создания и последнего изменения, имя владельца, а также краткую информацию о файлах. \\
\hline
\multicolumn{2}{ | l | }{Нефункциональные требования:} \\
\hline
ПСПН1 & Пользователю отображаются только те проекты, владельцем которых он является, или к которым он имеет доступ на чтение или редактирование .\\
\hline
\end{tabular}

\newpage

\item Создание проекта \\

\begin{tabular}{| p{2.5cm}  | p{11.5cm} |}
\hline
Описание & Создание проекта с указанием его названия и описания. \\
\hline
\multicolumn{2}{ | l | }{Функциональные требования:} \\
\hline
СПФ1 & При создании проекта система должна предоставить пользователю идентификатор, по которому он теперь сможет работать с только что созданным проектом. \\
\hline
СПФ2 & При создании проекта система предоставляет пользователю возможность ввести имя и описание нового проекта. \\
\hline
\end{tabular}

\item Управление правами доступа к проекту \\

\begin{tabular}{| p{2.5cm}  | p{11.5cm} |}
\hline
Описание & Предоставление доступа к проекту другим пользователям \\
\hline
\multicolumn{2}{ | l | }{Функциональные требования:} \\
\hline
ПДФ1 & Каждому пользователю можно выдать права доступа к проекту  \\
\hline
\multicolumn{2}{ | l | }{Нефункциональные требования:} \\
\hline
ПДН1 & Права пользователей подразделяются на чтение, редактирование. Права на чтение подразумевают только просмотр всех данных проекта и его изменений. Права на редактирование включают в себя права на чтение, а также возможность управлять жизненным циклом проекта. \\
\hline
ПДН2 & Только владелец проекта имеет возможность предоставлять какие-либо права доступа к проекту. \\
\hline
ПДН3 & По умолчанию новый проект доступен только его владельцу. \\
\hline
\end{tabular}

\item Изменение описательных данных(?) в проекте

\item Добавление в проект файлов с контентом, на основе которых и будет строиться конкретное решение архитектора \\
\begin{tabular}{| p{2.5cm}  | p{11.5cm} |}
\hline
\multicolumn{2}{ | c | }{В уже созданный проект происходит добавление нового файла с контентом. Загружаться данные могут как по ссылке, так и самим файлом с данными. Также файлы можно удалять.} \\
\hline
\multicolumn{2}{ | l | }{Функциональные требования:} \\
\hline
ДФФ1 & При невозможности загрузить данные система должна оповестить об этом пользователя (с указанием причины) \\
\hline
ДФФ2 & Для удаления файла из проекта система требует указание его идентификатора и повторное подтверждение запроса на удаление \\
\hline
ДФФ3 & При удалении файла из проекта система отображает этот файл только пользователям, имеющим права на редактирование \\
\hline
\multicolumn{2}{ | l | }{Требования к данным:} \\
\hline
ДФД1 & Формат загружаемых данных должен соответствовать стандарту IFC \\
\hline
ДФД2 & Максимальный размер загружаемых данных - 150 Мб \\
\hline
\end{tabular}

\item Изменение описательных данных(?) в файлах проекта

\item Просмотр контента текущей версии файла

\item Изменение контента файла \\
\begin{tabular}{| p{2.5cm}  | p{11.5cm} |}
\hline
\multicolumn{2}{ | c | }{Пользователь имеет возможность изменить контент неудаленных файлов в проектах.} \\
\hline
\multicolumn{2}{ | l | }{Функциональные требования:} \\
\hline
ВИФ1 & После внесения изменений в контент текущей версии файла система должна проверить корректность данного изменения и оповестить пользователя либо о невозможности выполнения, либо об успешности операции \\
\hline
ВИФ2 & После внесения изменений в контент файла система должна обновить историю проекта \\
\hline
\multicolumn{2}{ | l | }{Нефункциональные требования:} \\
\hline
ВИН1 & Вносить изменения разрешается только в неудаленные файлы \\
\hline
\end{tabular}

\item Просмотр списка изменений между различными версиями проекта

\item Отображение контента проекта или файла в определенный момент времени в истории

\item Отображение списка изменений описательных данных и контента проекта и файлов в каждой отдельно взятой единице записей изменений проекта

\item Просмотр конфликтов при изменении проекта

\item Разрешение конфликта при сохранении изменений в проект
\end {enumerate}


\begin{thebibliography}{1}
{\small
\bibitem{BIB_EXAMPLE} {\it Author1, Author2.}
\textbf{The name of example} // conference of this article. 2019. pp. 45-49
}
\end{thebibliography}

\end{document}