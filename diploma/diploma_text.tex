\documentclass[a4paper,14pt]{extreport} %размер бумаги устанавливаем А4, шрифт 14пунктов
\usepackage[T2A]{fontenc}
\usepackage{booktabs} % For prettier tables
\usepackage[utf8]{inputenc}%включаем свою кодировку: koi8-r или utf8 в UNIX, cp1251 в Windows
\usepackage[english,russian]{babel}%используем русский и английский языки с переносами
\usepackage{amssymb,amsfonts,amsmath,mathtext,cite,enumerate,float} %подключаем нужные пакеты расширений
\usepackage[dvips]{graphicx}
\usepackage{pgfplots}
%\usepackage{listings}
\usepackage[linesnumbered,boxed]{algorithm2e}
\graphicspath{{./images/}}%путь к рисункам


\makeatletter
\renewcommand{\@biblabel}[1]{#1.} % Заменяем библиографию с квадратных скобок на точку:
\makeatother

\usepackage{extsizes}

\usepackage{geometry} % Меняем поля страницы
\geometry{left=3cm}% левое поле
\geometry{right=2cm}% правое поле
\geometry{top=2cm}% верхнее поле
\geometry{bottom=2cm}% нижнее поле
\linespread{1.3}

%\renewcommand{\theenumi}{\arabic{enumi}}% Меняем везде перечисления на цифра.цифра
%\renewcommand{\labelenumi}{\arabic{enumi}}% Меняем везде перечисления на цифра.цифра
%\renewcommand{\theenumii}{.\arabic{enumii}}% Меняем везде перечисления на цифра.цифра
%\renewcommand{\labelenumii}{\arabic{enumi}.\arabic{enumii}.}% Меняем везде перечисления на цифра.цифра
%\renewcommand{\theenumiii}{.\arabic{enumiii}}% Меняем везде перечисления на цифра.цифра
%\renewcommand{\labelenumiii}{\arabic{enumi}.\arabic{enumii}.\arabic{enumiii}.}% Меняем везде перечисления на цифра.цифра


\begin{document}
\begin{titlepage}
\newpage

\begin{center}
\small МИНИСТЕРСТВО ОБРАЗОВАНИЯ И НАУКИ РОССИЙСКОЙ ФЕДЕРАЦИИ \\
\vspace{1cm}
\small ФЕДЕРАЛЬНОЕ ГОСУДАРСТВЕННОЕ АВТОНОМНОЕ ОБРАЗОВАТЕЛЬНОЕ \\*
\small УЧРЕЖДЕНИЕ ВЫСШЕГО ОБРАЗОВАНИЯ \\*
\small "МОСКОВСКИЙ ФИЗИКО-ТЕХНИЧЕСКИЙ ИНСТИТУТ \\*
\small (ГОСУДАРСТВЕННЫЙ УНИВЕРСИТЕТ)" \\*
\vspace{1cm}
\small ФАКУЛЬТЕТ ИННОВАЦИЙ И ВЫСОКИХ ТЕХНОЛОГИЙ \\*
\small КАФЕДРА ТЕОРЕТИЧЕСКИХ И ПРИКЛАДНЫХ ПРОБЛЕМ ИННОВАЦИЙ \\*
\hrulefill
\end{center}

\vspace{4em}

\begin{center}
\textbf{ВЫПУСКНАЯ КВАЛИФИКАЦИОННАЯ РАБОТА} \\
\vspace{1em}
\small \textbf{(МАГИСТЕРСКАЯ РАБОТА)} \\
\vspace{1em}
\small \textbf{Направление подготовки: 03.04.01 "Прикладные математика и физика"} \\
\vspace{1em}
\textsc{\textbf{НА ТЕМУ:}} \\
\vspace{2em}
\large \textsc{\textbf{Единая автоматизированная информационная система поддержки и сопровождения проектов, созданных с применением стандарта BIM}}
\end{center}

\vspace{6em}

\begin{flushleft}
Студент \hrulefill Княжев В.А. \\
\vspace{1em}
Научный руководитель \hrulefill Зырин С.В.\\
\vspace{1em}
%Рецензент \\
%к.ф.-м.н., в.н.с. АБВГ \hrulefill Петров В.В.\\
%\vspace{1.5em}
%Зав. кафедрой  ХХХ \\
%д.ф-м.н, профессор \hrulefill Сидоров Г.Г.
\end{flushleft}

\vspace{\fill}

\begin{center}
г. Москва, 2019
\end{center}

\end{titlepage}% это титульный лист
\tableofcontents % это оглавление, которое генерируется автоматически

\newpage
\chapter{Введение}
\section{Актуальность проблемы}

Темпы строительства зданий и промышленных объектов в мире и сложность конструкций увеличивается с каждым годом \cite{BUILDING_GROWTH_RATE}. Ранее использовавшиеся методы проектирования чертежей на бумаге отходят на второй план, и все более активно используются компьютерные технологии \cite{BUILDING_SOFTWARE}, а также становится очевидной необходимость повсеместного введения стандартов проектирования зданий. \\
Одним из наиболее современных стандартов проектирования является стандарт BIM (Building Information Modeling)\cite{BIM_FUTURE}. Его концепция позволяет не только проектировать здания, но также охватить весь их жизненный цикл: от управления затратами и строительством здания до его эксплуатации. \\
Подобная всеобъемлемость хороша тем, что вся информация о конструкции содержится в одном проекте. Это помогает сохранять целостность данных, позволяет быстрее выявлять ошибки и уменьшать стоимость ремонта. Но также из этого вытекает необходимость координации одновременной работы большого количества людей над одним проектом: крупных команд архитекторов, иногда распределенных по всему миру, эксплуатирующих организаций и всех других людей, участвующих в обслуживании здания. \\
Поэтому очень важно иметь возможность одновременного изменения BIM представления объекта разными людьми без потери каких-либо данных. Но малейшая ошибка в одном из элементов конструкции, не обнаруженная вовремя, может привести к серьезным последствиям, например к дополнительным затратам на проект. Поэтому важно в любой момент времени иметь доступ к электронному журналу аудита всех изменений проекта. \\
В настоящий момент программ, специализирующихся на архитектурных проектах стандарта BIM, и  которые бы в полной мере решали задачу по координации работы большого количества людей и отслеживания изменений, не существует. 

\newpage 
\section{Постановка задачи}

Требуется разработать веб-систему, которая бы могла предоставить пользователям следующие возможности:
\begin{enumerate}
\item Управление жизненным циклом проектов. \\
Создание проекта, добавление, редактирование  и удаление файлов, управление правами доступа к проекту.
\item Отслеживание изменений проекта во времени. \\
Отображение списка всех изменений проекта, а также возможность просмотра версии данных или внесенных в проект изменений в конкретный момент времени.
\item Одновременное внесение изменений в проекты несколькими пользователями. \\
Пользователи могут работать над разными частями проекта в одно и то же время. При наличии конфликтующих изменений предоставляется возможность сохранения изменений, внесенных как другими пользователями, так и текущим.
\item Подготовка окружения, запуск системы и ее масштабируемость. \\
Возможность быстрой подготовки окружения и запуска сервиса для мговенного развертывания веб-платформы. В моменты пиковой нагрузки пользователей, веб-платформа не должна терять производительность.
\end{enumerate}

\newpage

\chapter{Основная часть}
\section{Стандарт BIM}


\newpage
\section{Формат данных}


\newpage
\section{Пользовательские истории}


\newpage
\section{Бизнес-требования}

Бизнес-требования (business requirements) -- информация, в совокупности описывающая потребность, которая инициирует один или больше проектов с целью предоставить решение и получить требуемый конечный результат. В основу бизнес-требований ложатся бизнес-возможности, бизнес-цели, критерии успеха и положение о концепции. \\
Бизнес-требования определяют концепцию решения и границы проекта, в котором оно будет реализовываться. \\
Концепция и границы -- два базовых элемента бизнес-требований. \\ Концепция продукта (product vision) должна кратко описывать конечный продукт, который в свое время должен достигать заданных бизнес-целей. \\
Границы проекта (project scope) показывают, какая часть конечной концепции продукта будет реализована в текущей итерации. \\
В данной работе границы проекта совпадают с концепцией решения. \\
Документ о концепции и границах (vision and scope document) -- единый документ, который включает в себя все бизнес-требования. \\
Далее будут представлены основные пункты этого документа.

\subsection{Исходные данные}

На данный момент архитекторам требуется веб-платформа для одновременной работы с архитектурными проектами без потери данных, которая также предоставляла бы доступ к электронному журналу аудита всех изменений проектов.

\newpage
\subsection{Бизнес-цели}

\begin{table}[H]
\caption {Нефинансовые цели} \label{tab:title}
\begin{center}
\begin{tabular}{ | l | p{14cm} | }
\hline
№ & Цель \\
\hline
Н1 & Разработать веб-платформу для управления жизненным циклом	архитектурных проектов \\
\hline
Н2 & Реализовать возможность одновременного редактирования проектов и разрешения конфликтов в случаях их наличия \\
\hline
Н3 & Реализовать хранение журнала аудита всех изменений проектов и возможность его просмотра \\
\hline
\end{tabular}
\end{center}
\end{table}
 
\subsection{Критерии успеха}

\begin{itemize}
\item Веб-платформа позволяет управлять жизненным циклом архитектурного проекта.
\item Веб-платформе предоставляет возможность просмотра электронного журнала аудита изменений проекта.
\item Веб-платформа позволяет разрешать конфликты, возникающие при одновременном редактировании, без потери данных.
\end {itemize}
 
 \subsection{Положение о концепции проекта}
 
Для пользователей, которым требуется управлять жизненным циклом  архитектурных проектов и иметь возможность отслеживать изменения  во времени, данная работа является веб-платформой, которая будет выступать в качестве единой системы по хранению и изменению архитектурных проектов без потери данных с возможностью просмотра электронного журнала аудита изменений.

\newpage
\section{Ограничения системы}
\subsection{Основные функции}

\begin{enumerate}
\item Просмотр списка доступных пользователю проектов.
\item Создание проекта.
\item Управление правами доступа к проекту.
\item Добавление файлов в проект.
\item Изменение метадаты проекта и его файлов.
\item Удаление файла из проекта.
\item Просмотр контента файла.
\item Редактирование контента файла.
\item Просмотр журнала аудита изменений проекта.
\item Просмотр контента проекта в определенный промежуток времени.
\item Просмотр списка изменений, внесенных в проект в определенный момент времени.
\item Разрешение конфликтных ситуаций при редактировании файлов проекта.
\end {enumerate}

\subsection{Ограничения и исключения}

\begin{itemize}
\item Размер каждого файла должен не превышать 150 Мб (ограничение IFC формата).
\item В данной работе не предполагается возможность создания файлов со связанными между собой BIM представлениями объектов.
\end {itemize}

\newpage

\section{Функции системы}

\begin{enumerate}

\item Просмотр списка проектов

\begin{table}[H]
\caption {Просмотр списка проектов} \label{tab:title}
\begin{center}
\begin{tabular}{| p{2.5cm}  | p{11.5cm} |}
\hline
Описание & Пользователь может просмотреть список доступных ему преоктов. Также для поиска проектов имеется возможность фильтрации данных. \\
\hline
\multicolumn{2}{ | l | }{Функциональные требования:} \\
\hline
ПСПФ1 & Система должна предоставить список всех проектов по заданным фильтрам  \\
\hline
ПСПФ2 & Записи проектов должны содержать следующую информацию: имя, описание проекта, даты создания и последнего изменения, имя владельца, а также краткую информацию о файлах. \\
\hline
\multicolumn{2}{ | l | }{Нефункциональные требования:} \\
\hline
ПСПН1 & Пользователю отображаются только те проекты, владельцем которых он является, или к которым он имеет доступ на чтение или редактирование .\\
\hline
\end{tabular}
\end{center}
\end{table}

\newpage

\item Создание проекта \\

\begin{table}[H]
\caption {Создание проекта} \label{tab:title}
\begin{center}
\begin{tabular}{| p{2.5cm}  | p{11.5cm} |}
\hline
Описание & Создание проекта с указанием его названия и описания. \\
\hline
\multicolumn{2}{ | l | }{Функциональные требования:} \\
\hline
СПФ1 & При создании проекта система должна предоставить пользователю идентификатор, по которому он теперь сможет работать с только что созданным проектом. \\
\hline
СПФ2 & При создании проекта система предоставляет пользователю возможность ввести имя и описание нового проекта. \\
\hline
\end{tabular}
\end{center}
\end{table}

\item Управление правами доступа к проекту \\

\begin{table}[H]
\caption {Управление правами доступа} \label{tab:title}
\begin{center}
\begin{tabular}{| p{2.5cm}  | p{11.5cm} |}
\hline
Описание & Предоставление доступа к проекту другим пользователям \\
\hline
\multicolumn{2}{ | l | }{Функциональные требования:} \\
\hline
ПДФ1 & Каждому пользователю можно выдать права доступа к проекту  \\
\hline
\multicolumn{2}{ | l | }{Нефункциональные требования:} \\
\hline
ПДН1 & Права пользователей подразделяются на чтение, редактирование. Права на чтение подразумевают только просмотр всех данных проекта и его изменений. Права на редактирование включают в себя права на чтение, а также возможность управлять жизненным циклом проекта. \\
\hline
ПДН2 & Только владелец проекта имеет возможность предоставлять какие-либо права доступа к проекту. \\
\hline
ПДН3 & По умолчанию новый проект доступен только его владельцу. \\
\hline
\end{tabular}
\end{center}
\end{table}

\item Добавление файлов в проект \\

\begin{table}[H]
\caption {Добавление файлов в проект} \label{tab:title}
\begin{center}
\begin{tabular}{| p{2.5cm}  | p{11.5cm} |}
\hline
Описание & В уже созданный проект происходит добавление нового файла с контентом. Загружаться данные могут как по ссылке, так и самим файлом с данными. Также файлы можно удалять. \\
\hline
\multicolumn{2}{ | l | }{Функциональные требования:} \\
\hline
ДФФ1 & При невозможности загрузить данные система должна оповестить об этом пользователя (с указанием причины) \\
\hline
ДФФ2 & Для удаления файла из проекта система требует указание его идентификатора и повторное подтверждение запроса на удаление \\
\hline
ДФФ3 & При удалении файла из проекта система отображает этот файл только пользователям, имеющим права на редактирование \\
\hline
\multicolumn{2}{ | l | }{Требования к данным:} \\
\hline
ДФД1 & Формат загружаемых данных должен соответствовать стандарту IFC \\
\hline
ДФД2 & Максимальный размер загружаемых данных - 150 Мб \\
\hline
\end{tabular}
\end{center}
\end{table}

\item Редактирование контента файла \\

\begin{table}[H]
\caption {Редактирование контента файла} \label{tab:title}
\begin{center}
\begin{tabular}{| p{2.5cm}  | p{11.5cm} |}
\hline
Описание & Пользователь имеет возможность изменить контент неудаленных файлов в проектах. \\
\hline
\multicolumn{2}{ | l | }{Функциональные требования:} \\
\hline
ВИФ1 & После внесения изменений в контент текущей версии файла система должна проверить корректность данного изменения и оповестить пользователя либо о невозможности выполнения, либо об успешности операции \\
\hline
ВИФ2 & После внесения изменений в контент файла система должна обновить историю проекта \\
\hline
\multicolumn{2}{ | l | }{Нефункциональные требования:} \\
\hline
ВИН1 & Вносить изменения разрешается только в неудаленные файлы \\
\hline
\end{tabular}
\end{center}
\end{table}

\end {enumerate}


\newpage
\section{Описание системы}


\newpage
\section{Описание алгоритмов}


\newpage
\section{Инфраструктура веб-платформы}


\newpage
\section{Характеристики качества}


\newpage
\chapter{Заключение}
Kensek M. Karen; Noble, Douglas (2014). Building Information Modeling: BIM in Current and Future Practice (1st ed.).. — Hoboken, New Jersey: John Wiley..

\begin{thebibliography}{1}
{\small
\bibitem{BUILDING_GROWTH_RATE} {\it Author1, Author2.}
\textbf{The name of example} // conference of this article. 2019. pp. 45-49
\bibitem{BUILDING_SOFTWARE} {\it Author1, Author2.}
\textbf{The name of example} // conference of this article. 2019. pp. 45-49
\bibitem{BIM_FUTURE} {\it Karen M. Kensek, Douglas E. Noble.}
\textbf{Building Information Modeling: BIM in Current and Future Practice (1st ed.)} // 2014 Hoboken, New Jersey: John Wiley
\bibitem{BIB_EXAMPLE} {\it Author1, Author2.}
\textbf{The name of example} // conference of this article. 2019. pp. 45-49
}
\end{thebibliography}

\end{document}
