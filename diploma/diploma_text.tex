%\usepackage{extsizes}
\documentclass[14pt, a4paper, russian]{report}

\linespread{1.3}


\usepackage[centertags]{amsmath}
\usepackage{amsthm,amsfonts,amssymb}
\usepackage{indentfirst}

\usepackage{extsizes}
\usepackage[left=30mm,right=20mm,top=20mm,bottom=20mm,bindingoffset=0cm]{geometry}

\usepackage{cmap}
\usepackage{multirow}


% \usepackage{jmlda}
% \usepackage{amsmath}

\usepackage[T2A]{fontenc}
\usepackage[utf8]{inputenc}
\usepackage[english,russian]{babel}
%\usepackage{pscyr}
\RequirePackage{graphicx}
\RequirePackage{subfig}
\RequirePackage{epstopdf}   % for Mikhail Burmistrov burmisha@gmail.com
\RequirePackage{tikz}       % for Mikhail Burmistrov burmisha@gmail.com
\RequirePackage{pgfplots}

%\renewcommand{\rmdefault}{ftm}
%\frenchspacing

% Переопределение вставки графики
\newcounter{PictureNo}

\hyphenpenalty 100
\tolerance 10000

\bibliographystyle{unsrt}



\title{Единая автоматизированная информационная система поддержки и сопровождения проектов, созданных с применением стандарта BIM}
\author{Студент: Княжев В.А. \\ Научный руководитель: Зырин С. В.}


\begin{document}
\maketitle

\newpage

\tableofcontents
%\linenumbers
\chapter*{Введение}


\newpage

\chapter{Сравнение основных подходов к хранению данных}

\section{Сравнительный анализ  основных \\ характеристик SQL и NoSQL баз данных}

\begin{tabular}{ |p{4cm}|p{5cm}|p{5cm}|  }
	\hline
 	\textbf{Характеристика} & \textbf{SQL} & \textbf{NoSQL}\\
 	\hline
 	Язык запросов    & Единый структурированный язык & Меняется в зависимости от подхода к хранению данных \\
 	\hline
 	\multirow{3}{4cm}{Структура данных} & \multirow{3}{5cm}{Жестко заданная структура данных со связями} &Документы \\ \cline{3-3}
	& & Пары <key, value> \\ \cline{3-3}
	& & Графы \\\cline{3-3}
	\hline
	Масштабируемость & Вертикальная &  Горизонтальная \\
	\hline
	Транзакционность & Поддерживается &  Редкие случаи \\
 	\hline
\end{tabular}
\newline \newline
Остановимся на каждом пункте более подробно:
\begin{enumerate}
	\item Язык запросов: \\ РСУБД используют единый SQL-стандарт. Каждая NoSQL база данных реализует свой способ работы с данными.
	\item Структура данных: \\ РСУБД обычно используется для жестко заданных проработанных структур данных, которые не будут часто подвергаться изменениям. NoSQL же выделяется здесь своей возможностью хранения больших объёмов неструктурированной информации. Она не накладывает ограничений на типы хранимых данных. Более того, при необходимости в процессе работы можно добавлять новые типы данных.
	\item Масштабируемость: \\ Оба решения можно масштабировать вертикально (путём увеличения системных ресурсов). Однако, решения NoSQL обычно предоставляют простые способы горизонтального масштабирования (как пример, создание кластера из нескольких нод).
	\item Транзакционность: \\ Не все NoSQL решения имеют поддержку транзакций (одна из немногих - MongoDB). РСУБД же в своб очередь соответствую требованиям ACID (Atomicity, Consistency, Isolation, Durability — атомарность, непротиворечивость, изолированность, долговечность), что позволяет обеспечить целостность базы данных. 
\end{enumerate}

\section{Выбор способа хранения для сущностей проекта}

\begin{enumerate}
	\item \textbf{Project} \\
	Представляет мета информацию о проекте. Например, данные о времени и авторе проекта, название, описание проекта и схема хранения данных, на основе которой создан проект. То есть, можно с уверенностью сказать, что имеет жестко заданную структуру, котороая вряд ли будет меняться в ближайшее время. Для объектов данного типа можно использовать РСУБД. \\
	\item \textbf{Document} \\
	Представлен в виде древовидной структуры из констант или специализированных объектов. Сам файл выглядит как список блоков, каждый из которых имеет ссылки на вложенные в себя объекты. В данном случае можно либо хранить данные в документоориентированной бд и при обработке этих данных переводить в древовидную структуру, либо сразу преобразовывать документ в дерево и сохранять его узлы (имеют некоторую структуру, которая будет зависеть от количества параметров-объектов, от которых зависит наш узел) в графовую бд.
	\item \textbf{DomainEvent or Commit} \\
	Данная сущность имеет следующую структуру: метаинформация о родителях изменения, название ветки, в которой создано, время создания и автор данного изменения, а также набор самих точечных изменений в структуре документа. Можно хранить в РСУБД. При получении слепка проекта на конкретно заданном периоде времени достаточно будет просмотреть набор изменений (Commit) и поэтапно их применить к содержанию проекта (Document). В дальнейшем для этой сущности можно будет использовать распределенный реплицированный журнал фиксации изменений Kafka.
\end{enumerate}

\end{document}
